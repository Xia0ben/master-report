\chapter{Algorithms}

\label{efficient-opening-detection-commented}

\section{Efficient Opening Detection, Levihn M. and Stilman M. (2011), Commented}\label{eod_section}

\paragraph{Note on GET-$M'_{i}$-MATRIX}\label{get_mi_matrix_note} The world $W$ is represented by an occupancy grid: this procedure extends the obstacle $M_{i}$ by the robot's diameter, giving us $M'_{i}$, then represented as a binary matrix $M$, which has the size of the bounding box of $M'_{i}$.

% \paragraph{Note on absence of initially Blocking Areas}\label{no_ba_note} If there are no initially blocking areas, then it means there already are openings, and we must therefore return true. This limit case is not taken into account by the original algorithm formulation.

\paragraph{Note on GET-NEW-X/Y-POS}\label{x-y-pos_note} Simulate the set of manipulation actions $A_{M}$ and get the world coordinates of the object.

\paragraph{Note on interpreting the value of Z}\label{interpreting_z_note} If $Z$ is the 0-matrix, it means no new openings were detected, as all blocking areas still are blocking after the manipulation. Else, it means that one intersecting area has disappeared, meaning a possible new opening.

\paragraph{Note on detecting BAs}\label{check_blockage_note} Check for blockage between $M'_{i}$ (represented by $M$) and other obstacles (which data is contained in the occupancy grid $G$). For that we only call ASSIGN-NR if at least one of the two current elements of both matrices signals an obstacle ($\neq 0$).

\paragraph{Note on deletion}\label{deletion_note} If an index has already been deleted in an element of $BS$, delete it everywhere else because if part of a previous blocking area is detected, it means that the robot is still blocked by the same area. If for a same element of $BS$, there is an index in $BA[x][y] and BA^*_{s}[x][y]$, then it means that the blocking area still exists, thus we zero it in $BS$.

\begin{algorithm}[H]

  \caption{Efficient Local Opening Detection algorithm, Levihn et. al. (2011), commented}

  \label{alg:opening_detection}

  \begin{algorithmic}[1]

    \Procedure{CHECK-NEW-OPENING}{$G$, $M_{i}$, $A_{M}$, $BA$}

      \State $M \gets$ GET-$M'_{i}$-MATRIX($M_{i}$) \Comment{\nameref{get_mi_matrix_note}}
      \State $x_{offset} \gets M_{i}.x$ \Comment{$M_{i}.x$ and $M_{i}.y$ are the map coordinates of $M_{i}$'s top left corner.}
      \State $y_{offset} \gets M_{i}.y$
      \If{$BA$ is null} \Comment{BA needs not be recomputed if the environment did not change.}
        \State $BA \gets$ GET-BLOCKING-AREAS($x_{offset}$, $y_{offset}$, $M$, $G$) \Comment{Blocking areas before manip.}
      \EndIf
      \State $x_{offset} \gets$ GET-NEW-X-POS($A_{M}$, $M_{i}$) \Comment{\nameref{x-y-pos_note}}
      \State $y_{offset} \gets$ GET-NEW-Y-POS($A_{M}$, $M_{i}$)
      \State $BA_{s} \gets$ GET-BLOCKING-AREAS($x_{offset}$, $y_{offset}$, $M$, $G$) \Comment{Blocking areas after manip.}
      \State $BA^*_{s} \gets [0][0](dim(M))$ \Comment{Since the window of $BA_{s}$ shifted with the manipulation of $M_{i}$, ...}

      \For{$k$ from 0 to $|BA^*_{s}|$} \Comment{... we shift it back for the future comparison with $BA$.}
        \For{$l$ from 0 to $|BA^*_{s}[i]|$}
          \State $x \gets (x_{offset} - M_{i}.x) + k$
          \State $y \gets (y_{offset} - M_{i}.y) + l$
          \If{$0 < x < |BA^*_{s}|$ AND $0 < y < |BA^*_{s}[x]|$}
            \State $BA^*_{s}[x][y] \gets |BA^*_{s}|[k][l]$
          \EndIf
        \EndFor
      \EndFor

      \State $Z \gets$ COMPARE($BA$, $BA^*_{s}$) \Comment{Finally, compare the two blocking aread configurations.}
      \If{$Z = [0][0](dim(M))$} \Comment{\nameref{interpreting_z_note}}
        \State return $false$
      \EndIf
      \State return $true$

    \EndProcedure

    \\

    \Procedure{GET-BLOCKING-AREAS}{$x_{off}$, $y_{off}$, $M$, $G$}

    \State $index \gets 1$
    \State $BA \gets [0][0](dim(M))$ \Comment{$[0][0](dim(M))$ represents the 0-Matrix of dimensions = $dim(M)$}

    \For{$x$ from 0 to $|M|$} \Comment{Iterate over $M$ to detect and tag blocking areas.}
      \For{$y$ from 0 to $|M[x]|$}
        \If{$M[x][y] \neq 0$ AND $G[x+x_{off}][y+y_{off}] \neq 0$} \Comment{\nameref{check_blockage_note}}
          \State ASSIGN-NR($BA$, $x$, $y$, $index$)  \Comment{$BA$ and $index$ are directly modified in the call.}
        \EndIf
      \EndFor
    \EndFor

    \State return $BA$ \Comment{Return the saved information on the blocked areas.}

    \EndProcedure

    \algstore{opening_detection}

  \end{algorithmic}

\end{algorithm}


\clearpage

\begin{algorithm}[H]
  \begin{algorithmic}[1]
    \algrestore{opening_detection}

    \Procedure{ASSIGN-NR}{$BA$, $x$, $y$, $index$}

    \For{$i$ from -1 to 1} \Comment{Assignment is performed based on the 3*3 neighborhood.}
      \For{$j$ from -1 to 1}
        \If{$BA[x+i][y+j] \neq 0$} \Comment{If a number is already in the neighborhood, ...}
          \State $BA[x][y] \gets BA[x+i][y+j]$ \Comment{... the same number is assigned to the element.}
          \State \textbf{return}
        \EndIf
      \EndFor
    \EndFor

    \State $BA[x][y] \gets index$ \Comment{Else a new number is assigned, equal to the new index.}
    \State $index \gets index + 1$ \Comment{Only increment index if new intersection is created.}
    \State \textbf{return}

    \EndProcedure

    \\

    \Procedure{COMPARE}{$BA$, $BA^*_{s}$} \Comment{This function checks for non-zero entries in both matrices.}

    \State $BS \gets$ COPY($BA$)
    \State $del_{num} \gets \emptyset$ \Comment{Set of the indexes of obstacles to delete from $BS$.}

    \For{$x$ from 0 to $|BS|$} \Comment{Iterate over $BS$.}
      \For{$y$ from 0 to $|BS[x]|$}
        \If{$BA[x][y] \in del_{num}$} \Comment{\nameref{deletion_note}}
          \State $BS[x][y] \gets 0$
        \EndIf
        \If{$BA[x][y] \neq 0$ AND $BA^*_{s}[x][y] \neq 0$} \Comment{\nameref{deletion_note}}
          \State $del_{num} = del_{num} \bigcup BS[x][y]$
          \State $BS[x][y] \gets 0$
        \EndIf
      \EndFor
    \EndFor

    \State \textbf{return} $BS$

    \EndProcedure

  \end{algorithmic}

\end{algorithm}

