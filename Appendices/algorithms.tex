\chapter{Algorithms}\label{algorithms}

\section{Reworked Wu's algorithm (as desc. in \ref{removing_ambiguity_section})}\label{appendix_reworked_wu_section}

\begin{algorithm}[H]

  \caption{Optimized algorithm for NAMO in unknown environments of Wu et. al. (2010), fixed - MAIN LOOP}

  \label{alg:01-wu-optimized-part1}

  \begin{algorithmic}[1]

      \Procedure{OPTIMIZED}{$R_{init}$, $R_{goal}$}

        \State $R \gets R_{init}$ \label{lst:line:01_plan_execution_loop_1}

        \State $blockedObsL \gets \emptyset$

        \State $P_{sort} \gets \emptyset$

        \State $\mathcal{O}_{new} \gets \emptyset$

        \State $isManipSuccess \gets True$

        \State $I \gets empty\_occupation\_grid$

        \State $p_{opt}.components \gets$ [$A$*($R_{init}$, $R_{goal}$, $I.occGrid$)]

        \State $p_{opt}.cost \gets |p_{opt}.components[0]| * moveCost$

        \While{$R \neq R_{goal}$}

          \State UPDATE-FROM-NEW-INFORMATION($I$)\label{lst:line:update-from-new-information_note}

          \State $\mathcal{O}_{new} \gets \mathcal{O}_{new} \bigcup I.newObstacles$

          \State $isPlanNotColliding \gets p_{opt} \bigcap \mathcal{O}_{new} \neq \emptyset$ \label{lst:line:intersection_note}

          \If{\textbf{not}($isPlanNotColliding$ AND $isManipSuccess$)}

            \State $p_{opt}.components \gets$ [$A$*($R$, $R_{goal}$, $I.occGrid$)]
            \State $p_{opt}.cost \gets |p_{opt}.components[0]| * moveCost$ \label{lst:line:01_plan_execution_loop_2}

            \For{each $o \in \mathcal{O}_{new}$} \label{lst:line:01_obstacle evaluation_loop_1}
              \For{each possible push direction $d$ on $o$}
                \State $p \gets$ OPT-EVALUATE-ACTION($o$, $d$, $p_{opt}$, $I$, $R$, $R_{goal}$, $blockedObsL$)
                \If{$p \neq$ null}
                  \State $P_{sort}$.insert(p)
                \EndIf
              \EndFor
            \EndFor

            \If{$P_{sort} \neq \emptyset$}
              \State $p_{next} \gets P_{sort}[0]$
              \While{$p_{opt}.cost \geq p_{next}.minCost$ AND $p_{next} \neq$ null}
                \State $p \gets$ OPT-EVALUATE-ACTION($p_{next}.o$, $p_{next}.d$, $p_{opt}$, $I$, $R$, $R_{goal}$, $blockedObsL$) \label{lst:line:suppcondition}
                \If{$p \neq$ null AND $p.cost \leq p_{opt}.cost$}
                  \State $p_{opt} \gets p$
                \EndIf
                \State $p_{next} \gets P_{sort}$.getNext()
              \EndWhile
            \EndIf \label{lst:line:01_obstacle evaluation_loop_2}

            \algstore{wustore}

  \end{algorithmic}

\end{algorithm}

\begin{algorithm}[H]

  \label{alg:01-wu-optimized-part2}

  \begin{algorithmic}[1]

            \algrestore{wustore}

            \State $\mathcal{O}_{new} \gets \emptyset$ \label{lst:line:01_plan_execution_loop_3}

          \EndIf
          \If{$p_{opt}.components = \emptyset$} \label{lst:line:stop_condition_note}
            \State \textbf{return} $False$
          \EndIf

          \State $isManipSuccess \gets True$ \label{lst:line:plan_following_note_1}
          \State $R_{next} \gets p_{opt}$.getNextStep()
          \State $R_{real} \gets$ ROBOT-GOTO($R_{next}$)
          \If{$p_{opt}.nextStepComponent$ is $c_{2}$ AND $R_{real} \neq R_{next}$}
            \State $isManipSuccess \gets False$
            \State $blockedObsL \gets blockedObsL \bigcup p_{opt}.o$
          \EndIf
          \State $R \gets R_{real}$ \label{lst:line:plan_following_note_2}

        \EndWhile

        \State \textbf{return} $True$ \label{lst:line:01_plan_execution_loop_4}

    \EndProcedure

  \end{algorithmic}

\end{algorithm}


\begin{algorithm}[H]

  \caption{Optimized algorithm for NAMO in unknown environments of Wu et. al. (2010), fixed - ACTION EVALUATION SUBROUTINE}

  \label{alg:01-wu-optevaluateaction}

  \begin{algorithmic}[1]

    \Procedure{OPT-EVALUATE-ACTION}{$o$, $d$, $p_{opt}$, $I$, $R$, $R_{goal}$, $blockedObsL$}

      \If{$o \in blockedObsL$}
        \State \textbf{return} null
      \EndIf

      \State $P_{o,d} \gets \emptyset$
      \State $c_{1} \gets A$*($R$, $o.init$, $I.occGrid$) \Comment{\nameref{obstacle_pushpose_note}}
      \If{$c_{1} = \emptyset$} \Comment{\nameref{c1_note}}
        \State \textbf{return} null
      \EndIf
      \State $c_{2} \gets \emptyset$
      \State $oSimPose \gets o.init$

      \While{push on $o$ in $d$ possible AND $|c_{2}| * pushCost \leq p_{opt}.cost$} \Comment{\nameref{bound_note}}
        \State $oSimPose \gets oSimPose + one\_push\_in\_d$ \Comment{\nameref{push_in_d_note}}
        \If{push created new opening} \Comment{\nameref{opening_detection_note}}
          \State $c_{2} \gets \{o.init, oSimPose\}$ \Comment{\nameref{c2_note}}
          \State $c_{3} \gets A$*($oSimPose$, $R_{goal}$, $I.occGrid$)
          \If{$c_{3} \neq \emptyset$} \Comment{\nameref{c3_note}}
            \State $p.components \gets$ [$c_{1}$, $c_{2}$, $c_{3}$]
            \State $p.cost \gets (|c_{1}| + |c_{3}|) * moveCost + |c_{2}| * pushCost$
            \State $p.minCost \gets |c_{2}| * pushCost + |c_{3}| * moveCost$
            \State $p.o, p.d \gets o, d$
            \State $P_{o,d} \gets P_{o,d} \bigcup \{p\}$
          \EndIf
        \EndIf
      \EndWhile

      \State \textbf{return} $p \in P_{o,d}$ with minimal $p.cost$ or null if $P_{o,d} = \emptyset$

    \EndProcedure

  \end{algorithmic}

\end{algorithm}


\section{Interpretation of Levihn's recommandations (as desc. in \ref{levihn_pseudocode_section})}\label{appendix_levihn_interpretation_section}

\begin{algorithm}[H]

  \caption{Optimized algorithm for NAMO in unknown environments of Wu et. al. adapted according to M.Levihn et. al.'s (2014) recommandations - EXECUTION LOOP}

  \label{alg:02-levihn-makeandexecuteplan}

  \begin{algorithmic}[1]

    \Procedure{MAKE-AND-EXECUTE-PLAN}{$R_{init}$, $R_{goal}$}

      \State $R \gets R_{init}$
      \State $\mathcal{O}_{new} \gets \emptyset$
      \State $isManipSuccess \gets True$
      \State $blockedObsL \gets \emptyset$
      \State $euCostL, minCostL \gets \emptyset, \emptyset$
      \State $I \gets empty\_occupation\_grid$
      \State $p_{opt}.components \gets [D$*Lite($R_{init}$, $R_{goal}$, $I.occGrid$)] \Comment{\nameref{d_star_note}}
      \State $p_{opt}.cost \gets |p_{opt}| * moveCost$

      \While{$R \neq R_{goal}$}

        \State UPDATE-FROM-NEW-INFORMATION($I$) \Comment{\nameref{second_update-from-new-information_note}}

        \If{$I$.freeSpaceCreated()}
          \State $minCostL \gets \emptyset$
        \EndIf

        \State $\mathcal{O}_{new} \gets \mathcal{O}_{new} \bigcup I.newObstacles$

        \State $\mathcal{O} \gets I.allObstacles$

        \State $isPathFree \gets p_{opt} \bigcap \mathcal{O}_{new} \neq \emptyset$

        \If{\textbf{not}($isPathFree$ AND $isManipSuccess$)}
          \State $p_{opt}.components \gets [D$*Lite($R$, $R_{goal}$, $I.occGrid$)]
          \State $p_{opt}.cost \gets |p_{opt}| * moveCost$
          \State MAKE-PLAN($R$, $R_{goal}$, $I$, $\mathcal{O}$, $blockedObsL$, $p_{opt}$, $euCostL$, $minCostL$)
          \State $\mathcal{O}_{new} \gets \emptyset$
        \EndIf

        \If{$p_{opt}.components = \emptyset$}
          \State \textbf{return} $False$
        \EndIf

        \State $isManipSuccess \gets True$
        \State $R_{next} \gets p_{opt}$.getNextStep()
        \State $c_{next} \gets p_{opt}$.getNextStepComponent()
        \State $R_{real} \gets$ ROBOT-GOTO($R_{next}$)
        \If{$c_{next} = c_{2}$ AND $R_{real} \neq R_{next}$}
          \State $isManipSuccess \gets False$
          \State $blockedObsL \gets blockedObsL \bigcup p_{opt}.o$
        \EndIf
        \State $R \gets R_{real}$

        % \If{$c_{next} = c_{2}$ AND $R_{real} \neq R_{next}$}
        %   \State $isManipSuccess \gets False$
        %   \State $blockedObsL \gets blockedObsL \bigcup {p_{opt}.o, R_{real}, GET-SET-OF-POSSIBLY-BLOCKING-POINTS()}$ \Comment{The current pose $R_{real}$ required to push $o$ from where we left it is marked as "blocked" and will only be removed from the list if either the obstacle moves/is moved in a way that the saved pose is no longer corresponding to one of the $pushPoses$ of the obstacle or if the set of possibly blocking points is known as free.}
        % \EndIf

      \EndWhile

      \State \textbf{return} $True$

    \EndProcedure

  \end{algorithmic}
\end{algorithm}


\begin{algorithm}[H]

  \caption{Optimized algorithm for NAMO in unknown environments of Wu et. al. adapted according to M.Levihn et. al.'s (2014) recommandations - PLAN COMPUTATION}

  \label{alg:02-levihn-makeplan}

  \begin{algorithmic}[1]

    \Procedure{MAKE-PLAN}{$R$, $R_{goal}$, $I$, $\mathcal{O}$, $blockedObsL$, $p_{opt}$, $euCostL$, $minCostL$}

      \For{each $o \in \mathcal{O}$}
          \State $C_{3_{(Est)}} \gets \min(\{\forall graspPoint \in o.graspPoints \hspace{0.2cm} || \hspace{0.2cm} |\{graspPoint, R_{goal}\}|\})$
          \State $euCostL$.insertOrUpdate$(\{o, C_{3_{(Est)}}\})$
      \EndFor

      \State $i_{e}, i_{m} \gets 0 , 0$ \Comment{\nameref{get_list_element_note}}

      \While{$\min(minCostL[i_{m}].minCost, euCostL[i_{e}].c_{3_{est}}) < p_{opt}.cost$} \Comment{\nameref{list_traversal_note}}
        \If{$minCostL[i_{m}].minCost < euCostL[i_{e}].c_{3_{est}}$} \Comment{\nameref{minCostL_priority_note}}
          \State $p \gets$ PLAN-FOR-OBSTACLE($minCostL[i_{m}].obstacle$, $p_{opt}$, $I$, $R$, $R_{goal}$, $blockedObsL$)
          \If{$p \neq$ null}
            \State $minCostL.$insertOrUpdate$(\{minCostL[i_{m}].obstacle, p.minCost\})$
            \State $i_{m} \gets i_{m} + 1$
            \If{$p.cost < p_{opt}.cost$}
              \State $p_{opt} \gets p$
            \EndIf
          \Else
            \State $minCostL.$insertOrUpdate$(\{minCostL[i_{m}].obstacle, +\infty\})$
            \State $i_{m} \gets i_{m} + 1$
          \EndIf
        \Else
          \If{not $minCostL$.contains($euCostL[i_{e}].obstacle)$} \Comment{\nameref{postponing_note}}
            \State $p \gets$ PLAN-FOR-OBSTACLE($euCostL[i_{e}].obstacle$, $p_{opt}$, $I$, $R$, $R_{goal}$, $blockedObsL$)
            \If{$p \neq$ null}
              \State $minCostL.$insertOrUpdate$(\{euCostL[i_{e}].obstacle, p.minCost\})$
              \State $i_{m} \gets i_{m} + 1$
              \If{$p.cost < p_{opt}.cost$}
                \State $p_{opt} \gets p$
              \EndIf
            \Else \Comment{Corresponds to the "If $p \neq$ null" statement.}
              \State $minCostL.$insertOrUpdate$(\{euCostL[i_{e}].obstacle, +\infty\})$
              \State $i_{m} \gets i_{m} + 1$
            \EndIf
          \EndIf
          \State $i_{e} \gets i_{e} + 1$
        \EndIf
      \EndWhile
    \EndProcedure
  \end{algorithmic}
\end{algorithm}


\begin{algorithm}[H]

  \caption{Optimized algorithm for NAMO in unknown environments of Wu et. al. adapted according to M.Levihn et. al.'s (2014) recommandations - PLAN EVALUATION FOR A SINGLE OBSTACLE}

  \label{alg:02-levihn-planforobstacle}

  \begin{algorithmic}[1]

    \Procedure{PLAN-FOR-OBSTACLE}{$o$, $p_{opt}$, $I$, $R$, $R_{goal}$, $blockedObsL$}

      \If{$o \in blockedObsL$}
        \State \textbf{return} null
      \EndIf

      \State $P_{o,d}$ $\gets \emptyset$
      \State $c_{1} \gets D$*Lite$(R, o.init, I)$
      \If{$c_{1} = \emptyset$}
        \State \textbf{return} null
      \EndIf
      \State $BA \gets null$ \label{lst:line:remember_ba_note_1}

      \For{each possible manipulation direction $d$ on $o$}

        \State $seq \gets 1$

        \State $oSimPose \gets o.init + one\_translation\_in\_d$

        \State $c_{3_{(Est)}} \gets \{oSimPose, R_{goal}\}$ \label{lst:line:new_bound_1}

        \State $C_{est} \gets (|c_{1}| + |c_{3_{(Est)}}|) * moveCost + seq * |one\_translation\_in\_d| * pushCost$

        \While{$C_{est}$ $ \leq p_{opt}.cost$ AND manipulation on $o$ possible} \label{lst:line:new_bound_2}

          \If{CHECK-NEW-OPENING($I.occGrid$, $o$, $seq * one\_translation\_in\_d$, $BA$)} \label{lst:line:remember_ba_note_2}
            \State $c_{2} \gets \{o.init, oSimPose\}$
            \State $c_{3} \gets D$*Lite($oSimPose$, $R_{goal}$, $I.withSimulatedObstacleMove$)
            \If{$c_{3} \neq \emptyset$}
              \State $p.components \gets$ [$c_{1}$, $c_{2}$, $c_{3}$]
              \State $p.cost \gets (|c_{1}| + |c_{3}|) * moveCost + |c_{2}| * pushCost$
              \State $p.minCost \gets |c_{2}| * pushCost + |c_{3}| * moveCost$
              \State $p.o, p.d \gets o, d$
              \State $P_{o,d} \gets P_{o,d} \bigcup \{p\}$
              \If{$p.cost < p_{opt}.cost$}
                \State $p_{opt} \gets p$
              \EndIf
            \EndIf
          \EndIf

          \State $seq \gets seq + 1$

          \State $oSimPose \gets oSimPose + one\_translation\_in\_d$

          \State $c_{3_{(Est)}} \gets \{oSimPose, R_{goal}\}$

          \State $C_{est} \gets (|c_{1}| + |c_{3_{(Est)}}|) * moveCost + seq * |one\_translation\_in\_d| * pushCost$

        \EndWhile

      \EndFor

    \State \textbf{return} $p \in P_{o,d}$ with minimal $p.cost$ or null if $P_{o,d} = \emptyset$

    \EndProcedure

  \end{algorithmic}
\end{algorithm}


\section{Algorithm adapted to our use case (as desc. in \ref{discussion_hypotheses_section})}\label{appendix_basicmods_section}

\begin{algorithm}[H]

  \caption{Execution loop taking our hypotheses into account.}

  \label{alg:03-custom-basicmods-makeandexecuteplan}

  \begin{algorithmic}[1]

    \Procedure{MAKE-AND-EXECUTE-PLAN}{$R_{init}$, $R_{goal}$, $I_{init}$}

      \State \dots \label{lst:line:initbis1} \Comment{Initialization (lines \ref{lst:line:init1} to \ref{lst:line:init2} in Algorithm \ref{alg:02-levihn-makeandexecuteplan})}

      \State \hlgreen{$I \gets I_{init}$}

      \State $p_{opt}.components \gets [$\hlgreen{$A$*}($R_{init}$, $R_{goal}$, $I$)]
      \State $p_{opt}.cost \gets |p_{opt}| * moveCost$ \label{lst:line:initbis2}

      \While{$R \neq R_{goal}$}

        \State UPDATE-FROM-NEW-INFORMATION($I$) \label{lst:line:knowledge1}

        \If{$I$.freeSpaceCreated}
          \State $minCostL \gets \emptyset$
        \EndIf

        \State $\mathcal{O}_{new} \gets \mathcal{O}_{new} \bigcup I.newObstacles$

        \State $\mathcal{O} \gets I.allObstacles$

        \State $isPathFree \gets p_{opt} \bigcap \mathcal{O}_{new} \neq \emptyset$

        \State \hlgreen{$isPushPoseValid \gets True$} \label{lst:line:validity1}
        \State \hlgreen{$isManipSafe \gets True$}
        \State \hlgreen{$isObstacleSame \gets True$}
        \If{\hlgreen{$p_{opt}.o$ exists}} \Comment{If $p_{opt}$ includes the manipulation of an obstacle.}
          \State \hlgreen{$isObstacleSame \gets \mathcal{O}[p_{opt}.o] = p_{opt}.o$} \Comment{\nameref{operators_note}}
          \If{\hlgreen{\textbf{not} $isObstacleSame$}}
            \State \hlgreen{$p_{opt}.o \gets \mathcal{O}[p_{opt}.o.id]$} \Comment{Update the copy.}
            \State \hlgreen{$p_{opt}.safeSweptArea \gets$ GET-SAFE-SWEPT-AREA($p_{opt}.o$, $p_{opt}.translation$, $I$)}
            \If{\hlgreen{$p_{opt}.pushPose \not\in p_{opt}.o.pushPoses$}}
              \State \hlgreen{$isPushPoseValid \gets False$}
            \EndIf
          \EndIf
          \If{\hlgreen{$p_{opt}.safeSweptArea \bigcap \mathcal{O} \neq \emptyset$}} \Comment{\nameref{area_intersect_note}}
            \State \hlgreen{$isManipSafe \gets False$}
          \EndIf
        \EndIf \label{lst:line:knowledge2} \label{lst:line:validity2}

        \If{\textbf{not}($isPathFree$ AND $isManipSuccess$ \hlgreen{AND $isManipSafe$ AND $isPushPoseValid$})}
          \State $p_{opt}.components \gets$ [\hlgreen{$A$*}($R$, $R_{goal}$, $I$)]
          \State $p_{opt}.cost \gets |p_{opt}| * moveCost$
          \State MAKE-PLAN($R$, $R_{goal}$, $I$, $\mathcal{O}$, $blockedObsL$, $p_{opt}$, $euCostL$, $minCostL$)
          \State $\mathcal{O}_{new} \gets \emptyset$
        \EndIf

        \State \dots \Comment{Execution (lines \ref{lst:line:exec1} to \ref{lst:line:exec2} in Algorithm \ref{alg:02-levihn-makeandexecuteplan})}

      \EndWhile

      \State \textbf{return} $True$

    \EndProcedure

  \end{algorithmic}
\end{algorithm}


\begin{algorithm}[H]

  \caption{Obstacle evaluation subroutine taking our hypotheses into account.}

  \label{alg:03-custom-basicmods-planforobstacle}

  \begin{algorithmic}[1]

    \Procedure{PLAN-FOR-OBSTACLE}{$o$, $p_{opt}$, $I$, $R$, $R_{goal}$, $blockedObsL$}

      \If{$o \in blockedObsL$} \label{lst:line:initobs1}
        \State \textbf{return} null
      \EndIf \label{lst:line:initobs2}

      \State $P_{o,d}$ $\gets \emptyset$
      % \State $BA \gets null$

      \For{\hlgreen{each $pushPose$ in $o.pushPoses$}}
        \State \hlgreen{$pushUnit \gets (cos(pushPose.yaw), sin(pushPose.yaw))$} \label{lst:line:loop1} \Comment{Unit vector for push direction}

        \State \hlgreen{$c_{1} \gets A$*($R$, $pushPose$, $I$)} \Comment{$c_{1}$ is computed for each push pose}

        \If{$c_{1} = \emptyset$}
          \State \hlgreen{\textbf{continue}} \Comment{\nameref{continue_note}}
        \EndIf

        \State $seq \gets 1$

        \State \hlgreen{$translation \gets pushUnit * onePushDist * seq$} \Comment{onePushDist is a distance constant}

        \State \hlgreen{$safeSweptArea \gets $GET-SAFE-SWEPT-AREA($o$, $translation$, $I$)}

        \State $oSimPose \gets $\hlgreen{$pushPose + translation$} \label{lst:line:loop2}

        \State $c_{3_{(Est)}} \gets \{oSimPose, R_{goal}\}$

        \State $C_{est} \gets (|c_{1}| + |c_{3_{(Est)}}|) * moveCost + $\hlgreen{$|translation| * o.pushCost$}

        \While{$C_{est}$ $ \leq p_{opt}.cost$ AND \hlgreen{$safeSweptArea \neq$ null}}

          % \If{CHECK-NEW-OPENING($I.occGrid$, $o$, $translation$, $BA$)}
            \State $c_{2} \gets \{$\hlgreen{$pushPose$}$, oSimPose\}$
            \State \hlgreen{$c_{3} \gets A$*($oSimPose$, $R_{goal}$, $I.withSimulatedObstacleMove$)}
            \If{$c_{3} \neq \emptyset$}
              \State $p.components \gets$ [$c_{1}$, $c_{2}$, $c_{3}$]
              \State $p.cost \gets (|c_{1}| + |c_{3}|) * moveCost + |c_{2}| * $\hlgreen{$o.pushCost$}
              \State $p.minCost \gets |c_{2}| * $\hlgreen{$o.pushCost$}$ + |c_{3}| * moveCost$
              \State $p.o \gets$ \hlgreen{COPY($o$)} \label{lst:line:ovaraffec1} \Comment{\nameref{copy_note}}
              \State \hlgreen{$p.translation \gets translation$} \Comment{\nameref{translation_note}}
              \State \hlgreen{$p.safeSweptArea \gets safeSweptArea$}
              \State $P_{o,d} \gets P_{o,d} \bigcup \{p\}$
              \If{$p.cost < p_{opt}.cost$}
                \State $p_{opt} \gets p$
              \EndIf \label{lst:line:ovaraffec2}
            \EndIf
          % \EndIf

          \State $seq \gets seq + 1$ \label{lst:line:loopvarup1}

          \State \hlgreen{$translation \gets pushUnit * onePushDist * seq$}

          \State \hlgreen{$safeSweptArea \gets $GET-SAFE-SWEPT-AREA($o$, $translation$, $I$)}

          \State $oSimPose \gets $\hlgreen{$pushPose + translation$} \label{lst:line:loopvarup2}

          \State $c_{3_{(Est)}} \gets \{oSimPose, R_{goal}\}$

          \State $C_{est} \gets (|c_{1}| + |c_{3_{(Est)}}|) * moveCost + $\hlgreen{$|translation| * o.pushCost$}

        \EndWhile

      \EndFor

    \State \textbf{return} $p \in P_{o,d}$ with minimal $p.cost$ or null if $P_{o,d} = \emptyset$

    \EndProcedure

  \end{algorithmic}
\end{algorithm}


\section{Algorithm proposition: Social awareness through manipulation authorization consideration (as desc. in \ref{social_authorization_section})}\label{appendix_observation_section}

\begin{algorithm}[H]

  \caption{Execution loop modified for allowing observation.}

  \label{alg:04-custom-observation-makeandexecuteplan}

  \begin{algorithmic}[1]

    \Procedure{MAKE-AND-EXECUTE-PLAN}{$R_{init}$, $R_{goal}$, $I_{init}$}

      \State \dots \Comment{Initialization (lines 2 to 5 in Algorithm \ref{alg:03-custom-basicmods-makeandexecuteplan})}

      \State \hlgreen{$isObservable \gets True$}

      \While{$R \neq R_{goal}$}

        \State UPDATE-FROM-NEW-INFORMATION($I$)

        \State \dots \Comment{Knowledge update and checks (lines 7 to 29 in Algorithm \ref{alg:03-custom-basicmods-makeandexecuteplan})}

        \If{\textbf{not}($isPathFree$ AND $isManipSuccess$ AND $isManipSafe$ AND $isPushPoseValid$ \hlgreen{AND $isObservable$})}
          \State $p_{opt}.components \gets [A$*($R$, $R_{goal}$, $I.occGrid$)]
          \State $p_{opt}.cost \gets |p_{opt}| * moveCost$
          \State MAKE-PLAN($R$, $R_{goal}$, $I$, $\mathcal{O}$, $blockedObsL$, $p_{opt}$, $euCostL$, $minCostL$)
          \State $\mathcal{O}_{new} \gets \emptyset$
        \EndIf

        \If{$p_{opt}.components = \emptyset$}
          \State \textbf{return} $False$
        \EndIf

        \State $isManipSuccess \gets True$
        \State \hlgreen{$isObservable \gets True$}
        \State $R_{next} \gets p_{opt}$.getNextStep()
        \State $c_{next} \gets p_{opt}$.getNextStepComponent()
        \If{\hlgreen{$p_{opt}.o.movableStatus = IS\_MAYBE\_MOVABLE$ AND \textbf{not} $isObstacleSame$ AND ($c_{next} = o_{1}$ OR $c_{next} = c_{1}$)}}
          \State \hlgreen{$fObsPose \gets$ GET-FIRST-PATH-OBSPOSE($p_{opt}.o$, $p_{opt}$.get-$o_{1}$() + $p_{opt}$.get-$c_{1}$(), $I$)}
          \If{\hlgreen{$fObsPose =$ null}}
            \State \hlgreen{$isObservable \gets False$}
            \State \hlgreen{\textbf{continue}}
          \EndIf
        \EndIf
        \State $R_{real} \gets$ ROBOT-GOTO($R_{next}$)
        \If{$c_{next} = c_{2}$ AND $R_{real} \neq R_{next}$}
          \State $isManipSuccess \gets False$
          \State $blockedObsL \gets blockedObsL \bigcup p_{opt}.o$
        \EndIf
        \State $R \gets R_{real}$

      \EndWhile

      \State \textbf{return} $True$

    \EndProcedure

  \end{algorithmic}
\end{algorithm}


\begin{algorithm}[H]

  \caption{Obstacle evaluation subroutine modified for allowing observation.}

  \label{alg:04-custom-observation-planforobstacle}

  \begin{algorithmic}[1]

    \Procedure{PLAN-FOR-OBSTACLE}{$o$, $p_{opt}$, $I$, $R$, $R_{goal}$, $blockedObsL$}

      \If{$o \in blockedObsL$ \hlgreen{OR $o.movableStatus = IS\_NOT\_MOVABLE$}}
        \State \textbf{return} null
      \EndIf

      \State $P_{o,d}$ $\gets \emptyset$
      % \State $BA \gets null$

      \For{each $pushPose$ in $o.pushPoses$}
        \State $pushUnit \gets (cos(pushPose.yaw), sin(pushPose.yaw))$

        \State $c_{1} \gets A$*($R$, $pushPose$, $I$)

        \If{$c_{1} = \emptyset$}
          \State \textbf{continue}
        \EndIf

        %% BEGIN HLGREEN
        \State \hlgreen{$c_{0} \gets \emptyset$}

        \If{\hlgreen{$o.movableStatus = IS\_MAYBE\_MOVABLE$}}

          \State \hlgreen{$obsPose \gets$ GET-FIRST-PATH-OBSPOSE($o$, $c_{1}$, $I$)}

          \If{\hlgreen{$obsPose \neq$ null}}
            \State \hlgreen{$c_{0}, c_{1} \gets c_{1}[c_{1}.firstPose:obsPose], c_{1}[obsPose:c_{1}.lastPose]$}
          \Else
            \State \hlgreen{COMPUTE-O1-C1($o$, $I$, $R$, $pushPose$, $c_{0}$, $c_{1}$)}

            \If{\hlgreen{$c_{0} = \emptyset$ OR $c_{1} = \emptyset$}}
              \State \hlgreen{\textbf{continue}}
            \EndIf
          \EndIf
        \EndIf
        %% STOP HLGREEN

        \State $seq \gets 1$

        \State $translation \gets pushUnit * onePushDist * seq$

        \State $safeSweptArea \gets $GET-SAFE-SWEPT-AREA($o$, $translation$, $I$)

        \State $oSimPose \gets pushPose + translation$

        \State $c_{3_{(Est)}} \gets \{oSimPose, R_{goal}\}$

        \State $C_{est} \gets ($\hlgreen{$(c_{0} \neq \emptyset ? |c_{0}| : 0)$}$ + |c_{1}| + |c_{3_{(Est)}}|) * moveCost + |translation| * o.pushCost$

        \While{$C_{est}$ $ \leq p_{opt}.cost$ AND $safeSweptArea \neq$ null}

          % \If{CHECK-NEW-OPENING($I.occGrid$, $o$, $translation$, $BA$)}
            \State $c_{2} \gets \{pushPose, oSimPose\}$
            \State $c_{3} \gets A$*($oSimPose$, $R_{goal}$, $I.withSimulatedObstacleMove$)
            \If{$c_{3} \neq \emptyset$}
              \State $p.components \gets$ \hlgreen{$c_{0} \neq \emptyset$ ? [$c_{0}$, $c_{1}$, $c_{2}$, $c_{3}$] : [$c_{1}$, $c_{2}$, $c_{3}$]}
              \State $p.cost \gets ($\hlgreen{$(c_{0} \neq \emptyset ? |c_{0}| : 0)$}$ + |c_{1}| + |c_{3}|) * moveCost + |c_{2}| * o.pushCost$
              \State $p.minCost \gets |c_{2}| * o.pushCost + |c_{3}| * moveCost$
              \State \dots \Comment{Affectation of other variables of $p$ (lines \ref{lst:line:ovaraffec1} to \ref{lst:line:ovaraffec2} in Algorithm \ref{alg:03-custom-basicmods-planforobstacle})}
            \EndIf
          % \EndIf

          \State $seq \gets seq + 1$

          \State $translation \gets pushUnit * onePushDist * seq$

          \State $safeSweptArea \gets $GET-SAFE-SWEPT-AREA($o$, $translation$, $I$)

          \State $oSimPose \gets pushPose + translation$

          \State $c_{3_{(Est)}} \gets \{oSimPose, R_{goal}\}$

          \State $C_{est} \gets ($\hlgreen{$(c_{0} \neq \emptyset ? |c_{0}| : 0)$}$ + |c_{1}| + |c_{3_{(Est)}}|) * moveCost + |translation| * o.pushCost$

        \EndWhile

      \EndFor

    \State \textbf{return} $p \in P_{o,d}$ with minimal $p.cost$ or null if $P_{o,d} = \emptyset$

    \EndProcedure

  \end{algorithmic}
\end{algorithm}

\begin{algorithm}[H]

  \caption{Subroutine for getting the first pose in a $path$ that allows identification of $o$, it it exists.}

  \label{alg:04-custom-observation-simple-checkpath}
  
  \begin{algorithmic}[1]

    \Procedure{GET-FIRST-PATH-OBSPOSE}{$o$, $path$, $I$}
      \For{each $pose$ in $path$}
        \If{IS-OBS-IN-FOV-FOR-POSE($o$, $pose$, $I$)}
          \State \textbf{return} $pose$
        \EndIf
      \EndFor
      \State \textbf{return} null
    \EndProcedure
  \end{algorithmic}
\end{algorithm}

\begin{algorithm}[H]

  \caption{Subroutine for computing $c_{0}$ and $c_{1}$ if $c_{1}$ is not already valid.}

  \label{alg:04-custom-observation-simple-compute01c1}

  \begin{algorithmic}[1]

    \Procedure{COMPUTE-C0-C1}{$o$, $I$, $R$, $pushPose$, $c_{0}$, $c_{1}$}
      \State $c_{1} \gets \emptyset$
      \State $totalCost \gets +\infty$

      \For{each $obsPose$ in $o.obsPoses$}
        \State $o \gets A$*($R$, $obsPose$, $I$)
        \State $c \gets A$*($obsPose$, $pushPose$, $I$)
        \State $newTotalCost = |o| + |c|$
        \If{$newTotalCost < +\infty$ AND ($newTotalCost < totalCost$ OR ($newTotalCost = totalCost$ AND $|o| < |c_{0}|$)}
          \State $c_{0} = o$
          \State $c_{1} = c$
          \State $totalCost = |c_{0}| + |c_{1}|$
        \EndIf
      \EndFor
    \EndProcedure

  \end{algorithmic}
\end{algorithm}


\begin{algorithm}[H]

  \caption{Optimized subroutine for computing $c_{0}$ and $c_{1}$ if $c_{1}$ is not already valid.}

  \label{alg:04-custom-observation-optimized-compute01c1}

  \begin{algorithmic}[1]

    \Procedure{OPT-COMPUTE-C0-C1}{$o$, $I$, $R$, $pushPose$, $c_{0}$, $c_{1}$}
      \State $c_{1} \gets \emptyset$
      \State $totalCost \gets +\infty$

      %   % The computation time could be reduced by computing the heuristic
      %   % euclidean cost from current position to obstacle only once and not
      %   % as many times as there are push poses for the obstacle.
      %   % This needs to be added to the calling method and euCostToObsL passed in parameter.
      % \If{$o.movableStatus = IS\_MAYBE\_MOVABLE$}
      %   \State $euCostToObsL \gets \emptyset$
      %   \For{each $obsPose$ in $o.obsPoses$}
      %     \State $euCostToObsL$.insert$(\{obsPose, |{R, obsPose}|\})$
      %   \EndFor
      %
      % \EndIf

      \State $euPosesCostL \gets \emptyset$ \Comment{Sort observation poses by ascending heuristic cost.} \label{lst:line:occ_heur_1}
      \For{each $obsPose$ in $o.obsPoses$}
        \State $euPosesCostL$.insert$(\{obsPose, |{R, obsPose}| + |{obsPose, pushPose}|\})$
      \EndFor \label{lst:line:occ_heur_2}

      \If{$euPosesCostL \neq \emptyset$} \label{lst:line:traverse_eu_1}
        \State $op_{next} \gets euPosesCostL[0]$
        \While{$totalCost \geq op_{next}.cost$ AND $op_{next} \neq$ null}
          \State $c_{0_{cur}} \gets A$*($R$, $op_{next}.obsPose$, $I.occGrid$)
          \State $c_{1_{cur}} \gets A$*($op_{next}.obsPose$, $pushPose$, $I.occGrid$)
          \State $newTotalCost = |c_{0_{cur}}| + |c_{1_{cur}}|$
          \If{$newTotalCost < +\infty$ AND ($newTotalCost < totalCost$ OR ($newTotalCost = totalCost$ AND $|c_{0_{cur}}| < |c_{0}|$)}
            \State $c_{0} = c_{0_{cur}}$
            \State $c_{1} = c_{1_{cur}}$
            \State $totalCost = |c_{0}| + |c_{1}|$
          \EndIf
          \State $op_{next} \gets euPosesCostL$.getNext()
        \EndWhile
      \EndIf \label{lst:line:traverse_eu_2}
    \EndProcedure

  \end{algorithmic}
\end{algorithm}


\section{Algorithm proposition: Social awareness through placement consideration (as desc. in \ref{social_appendix_placement_section})}\label{appendix_placement_section}

\begin{algorithm}[H]

  \caption{Obstacle evaluation subroutine modified for considering placement.}

  \label{alg:05-custom-placement-planforobstacle}

  \begin{algorithmic}[1]

    \Procedure{PLAN-FOR-OBSTACLE}{$o$, $p_{opt}$, $I$, $R$, $R_{goal}$, $blockedObsL$, $occCostGrid$}

      \State \dots \Comment{Initialization (lines \ref{lst:line:initobs1} to \ref{lst:line:initobs2} in Algorithm \ref{alg:03-custom-basicmods-planforobstacle})}

      \For{each $pushPose$ in $o.pushPoses$}
        \State \dots \Comment{Loop initialization (lines \ref{lst:line:loop1} to \ref{lst:line:loop2} in Algorithm \ref{alg:03-custom-basicmods-planforobstacle})}

        \State \hlgreen{$suppC_{M} \gets $GET-OCC-COST(GET-OBS-POINTS($o$, $translation$), $occCostGrid$)}

        \State $c_{3_{(Est)}} \gets \{oSimPose, R_{goal}\}$

        \State $C_{est} \gets (|c_{1}| + |c_{3_{(Est)}}|) * moveCost + |translation| * o.pushCost *$ \hlgreen{$suppC_{M}$}

        \While{$C_{est}$ $ \leq p_{opt}.cost$ AND $safeSweptArea \neq$ null}

          % \If{CHECK-NEW-OPENING($I.occGrid$, $o$, $translation$, $BA$)}
            \State $c_{2} \gets \{pushPose, oSimPose\}$
            \State $c_{3} \gets A$*($oSimPose$, $R_{goal}$, $I.withSimulatedObstacleMove$)
            \If{$c_{3} \neq \emptyset$}
              \State $p.components \gets$ [$c_{1}$, $c_{2}$, $c_{3}$]
              \State $p.cost \gets (|c_{1}| + |c_{3}|) * moveCost + |c_{2}| * o.pushCost *$ \hlgreen{$suppC_{M}$}
              \State $p.minCost \gets |c_{2}| * o.pushCost *$ \hlgreen{$suppC_{M}$} $+ |c_{3}| * moveCost$
              \State \dots \Comment{Affectation of other variables of $p$ (lines \ref{lst:line:ovaraffec1} to \ref{lst:line:ovaraffec2} in Algorithm \ref{alg:03-custom-basicmods-planforobstacle})}
            \EndIf
          % \EndIf

          \State \dots \Comment{Loop variable update (lines \ref{lst:line:loopvarup1} to \ref{lst:line:loopvarup2} in Algorithm \ref{alg:03-custom-basicmods-planforobstacle})}

          \State \hlgreen{$suppC_{M} \gets $GET-OCC-COST(GET-OBS-POINTS($o$, $translation$), $occCostGrid$)}

          \State $c_{3_{(Est)}} \gets \{oSimPose, R_{goal}\}$

          \State $C_{est} \gets (|c_{1}| + |c_{3_{(Est)}}|) * moveCost + |translation| * o.pushCost *$ \hlgreen{$suppC_{M}$}

        \EndWhile

      \EndFor

    \State \textbf{return} $p \in P_{o,d}$ with minimal $p.cost$ or null if $P_{o,d} = \emptyset$

    \EndProcedure

  \end{algorithmic}
\end{algorithm}

\begin{algorithm}[H]

  \caption{Obstacle evaluation subroutine modified for considering placement.}

  \label{alg:05-custom-placement-getocccost}

  \begin{algorithmic}[1]

    \Procedure{GET-OCC-COST}{$simOccPoints$, $occCostGrid$}

      \State{$VALUE\_RANGE \gets (FORBIDDEN\_VALUE - ALLOWED\_VALUE)$}

      \State{$cost \gets 1$}

      \For{each $point$ in $simOccPoints$}
        \State $valueForPoint \gets occCostGrid[point]$
        \If{$valueForPoint = FORBIDDEN\_VALUE$} \label{lst:line:maxocccost_1}
          \State \textbf{return} $+\infty$ \label{lst:line:maxocccost_2}
        \ElsIf{$valueForPoint = ALLOWED\_VALUE$} \label{lst:line:noocccost_1}
          \State $valueForPoint \gets 0$
        \EndIf \label{lst:line:noocccost_2}
        \State{$cost \gets cost + valueForPoint / VALUE\_RANGE$} \label{lst:line:sumocccost}
      \EndFor

      \State \textbf{return} $cost$
    \EndProcedure

  \end{algorithmic}
\end{algorithm}


\section{Algorithm proposition: Taking dynamic obstacles into account (as desc. in \ref{dynamic_section})}\label{appendix_dynamic_section}

\begin{algorithm}[H]

  \caption{Execution loop taking dynamic obstacles into account.}

  \label{alg:06-custom-dynamic-makeandexecuteplan}

  \begin{algorithmic}[1]

    \Procedure{MAKE-AND-EXECUTE-PLAN}{$R_{init}$, $R_{goal}$, $I_{init}$}

      \State \dots \Comment{Initialization (lines \ref{lst:line:initbis1} to \ref{lst:line:initbis2} in Algorithm \ref{alg:03-custom-basicmods-makeandexecuteplan})}

      \State $isBlockingObsMoved \gets False$

      \While{$R \neq R_{goal}$}

        \State UPDATE-FROM-NEW-INFORMATION($I$)

        \If{$I$.freeSpaceCreated}
          \State $minCostL \gets \emptyset$
        \EndIf

        \State $\mathcal{O} \gets I.allObstacles$

        \State $isPathFree \gets p_{opt} \bigcap \mathcal{O} \neq \emptyset$

        \State \hlgreen{$isBlockingObsMoved \gets I.movedObstacles \neq \emptyset$}

        \State \dots \Comment{Plan validity checks (lines \ref{lst:line:validity1} to \ref{lst:line:validity2} in Algorithm \ref{alg:03-custom-basicmods-makeandexecuteplan})}

        \State \hlgreen{$isPlanValid \gets$ ($isPathFree$ AND $isManipSuccess$ AND $isManipSafe$ AND $isPushPoseValid$)}

        \If{\hlgreen{\textbf{not} $isPlanValid$}}
          \State $p_{opt}.components \gets$ [$A$*($R$, $R_{goal}$, $I$)]
          \State $p_{opt}.cost \gets |p_{opt}| * moveCost$
        \EndIf
        \If{\hlgreen{\textbf{not} $isPlanValid$ OR ($isBlockingObsMoved$)}}
          \State MAKE-PLAN($R$, $R_{goal}$, $I$, $\mathcal{O}$, $blockedObsL$, $p_{opt}$, $euCostL$, $minCostL$)
          \State \hlgreen{$isManipSuccess \gets True$} \Comment{Line \ref{lst:line:ismanipsuccess} in Algorithm \ref{alg:02-levihn-makeandexecuteplan} is moved here.}
          \State \hlgreen{\textbf{continue}}
        \EndIf

        \State \dots \Comment{Execution (lines \ref{lst:line:exec1} to \ref{lst:line:exec1bis} and \ref{lst:line:exec2bis} to \ref{lst:line:exec2} in Algorithm \ref{alg:02-levihn-makeandexecuteplan})}
        \State \dots \Comment{Execution (lines \ref{lst:line:exec1} to \ref{lst:line:exec2} in Algorithm \ref{alg:02-levihn-makeandexecuteplan})}

      \EndWhile

      \State \textbf{return} $True$

    \EndProcedure

  \end{algorithmic}
\end{algorithm}


\section{Merged proposition algorithm (as desc. in \ref{merged_proposition_section})}\label{appendix_merged_proposition_section}

\begin{algorithm}[H]

  \caption{Merged execution loop.}

  \label{alg:07-custom-merge-makeandexecuteplan-part1}

  \begin{algorithmic}[1]

    \Procedure{MAKE-AND-EXECUTE-PLAN}{$R_{init}$, $R_{goal}$, $I_{init}$}

      \State $R \gets R_{init}$
      \State $\mathcal{O}_{new} \gets \emptyset$
      \State $isManipSuccess \gets True$
      \State $blockedObsL \gets \emptyset$
      \State $euCostL, minCostL \gets \emptyset, \emptyset$

      \State $I \gets I_{init}$

      \State $p_{opt}.components \gets [$$A$*($R_{init}$, $R_{goal}$, $I.occGrid$)]
      \State $p_{opt}.cost \gets |p_{opt}| * moveCost$

      \State $isObservable \gets True$

      \While{$R \neq R_{goal}$}

        \State UPDATE-FROM-NEW-INFORMATION($I$)

        \If{$I$.freeSpaceCreated}
          \State $minCostL \gets \emptyset$
        \EndIf

        \State $\mathcal{O}_{new} \gets \mathcal{O}_{new} \bigcup I.newObstacles$

        \State $\mathcal{O} \gets I.allObstacles$

        \State $isPlanNotColliding \gets p_{opt} \bigcap \mathcal{O}_{new} \neq \emptyset$

        \State $isBlockingObsMoved \gets I.movedObstacles \neq \emptyset$

        \State $isPushPoseValid \gets True$
        \State $isManipSafe \gets True$
        \State $isObstacleSame \gets True$
        \If{$p_{opt}.o$ exists}
          \State $isObstacleSame \gets \mathcal{O}[p_{opt}.o] = p_{opt}.o$
          \If{\textbf{not} $isObstacleSame$}
            \State $p_{opt}.o \gets \mathcal{O}[p_{opt}.o.id]$
            \State $p_{opt}.safeSweptArea \gets$ GET-SAFE-SWEPT-AREA($p_{opt}.o$, $p_{opt}.translation$, $I$)
            \If{$p_{opt}.nextStepComponent$ is $ c_{1}$ AND $p_{opt}.pushPose \not\in p_{opt}.o.pushPoses$}
              \State $isPushPoseValid \gets False$
            \EndIf
          \EndIf
          \If{$p_{opt}.safeSweptArea \bigcap \mathcal{O} \neq \emptyset$}
            \State $isManipSafe \gets False$
          \EndIf
        \EndIf

        \State $isPlanValid \gets$ ($isPlanNotColliding$ AND $isManipSuccess$ AND $isManipSafe$ AND $isPushPoseValid$ AND $isObservable$)

        \If{\textbf{not} $isPlanValid$}
          \State $p_{opt}.components \gets$ [$A$*($R$, $R_{goal}$, $I.occGrid$)]
          \State $p_{opt}.cost \gets |p_{opt}| * moveCost$
        \EndIf
        \If{\textbf{not} $isPlanValid$ OR ($isBlockingObsMoved$)}
          \State MAKE-PLAN($R$, $R_{goal}$, $I$, $\mathcal{O}$, $blockedObsL$, $p_{opt}$, $euCostL$, $minCostL$)
          \State $isManipSuccess \gets True$
          \State \textbf{continue}
        \EndIf

        \algstore{makeandexecuteplan_merge_store}

  \end{algorithmic}
\end{algorithm}

\begin{algorithm}[H]

  \label{alg:07-custom-merge-makeandexecuteplan-part2}

  \begin{algorithmic}[1]

        \algrestore{makeandexecuteplan_merge_store}

        \If{$p_{opt}.components = \emptyset$}
          \State \textbf{return} $False$
        \EndIf

        \State $isManipSuccess \gets True$
        \State $isObservable \gets True$
        \State $R_{next} \gets p_{opt}$.getNextStep()
        \If{$p_{opt}.o \neq$ null $p_{opt}.o.movableStatus = IS\_MAYBE\_MOVABLE$ AND \textbf{not} $isObstacleSame$ AND ($c_{next} = c_{0}$ OR $c_{next} = c_{1}$)}
          \State $fObsPose \gets$ GET-FIRST-PATH-OBSPOSE($p_{opt}.o$, $p_{opt}$.get-$c_{0}$() + $p_{opt}$.get-$c_{1}$(), $I$)
          \If{$fObsPose =$ null}
            \State $isObservable \gets False$
            \State \textbf{continue}
          \EndIf
        \EndIf
        \State $R_{real} \gets$ ROBOT-GOTO($R_{next}$)
        \If{$p_{opt}.nextStepComponent$ is $c_{2}$ AND $R_{real} \neq R_{next}$}
          \State $isManipSuccess \gets False$
          \State $blockedObsL \gets blockedObsL \bigcup p_{opt}.o$
        \EndIf
        \State $R \gets R_{real}$

      \EndWhile

      \State \textbf{return} $True$

    \EndProcedure

  \end{algorithmic}
\end{algorithm}


\begin{algorithm}[H]

  \caption{Merged obstacle evaluation subroutine}

  \label{alg:07-custom-merge-optimized-planforobstacle-part1}

  \begin{algorithmic}[1]

    \Procedure{PLAN-FOR-OBSTACLE}{$o$, $p_{opt}$, $I$, $R$, $R_{goal}$, $blockedObsL$}

      \If{$o \in blockedObsL$ OR $o.movableStatus = IS\_NOT\_MOVABLE$}
        \State \textbf{return} null
      \EndIf

      \State $P_{o,d}$ $\gets \emptyset$
      % \State $BA \gets null$

      \For{each $pushPose$ in $o.pushPoses$}
        \State $pushUnit \gets (cos(pushPose.yaw), sin(pushPose.yaw))$

        \State $c_{1} \gets A$*($R$, $pushPose$, $I$)

        \If{$c_{1} = \emptyset$}
          \State \textbf{continue}
        \EndIf

        \State $c_{0} \gets \emptyset$

        \If{$o.movableStatus = IS\_MAYBE\_MOVABLE$}

          \State $obsPose \gets$ GET-FIRST-PATH-OBSPOSE($o$, $c_{1}$, $I$)

          \If{$obsPose \neq$ null}
            \State $c_{0}, c_{1} \gets c_{1}[c_{1}.firstPose:obsPose], c_{1}[obsPose:c_{1}.lastPose]$
          \Else
            \State OPT-COMPUTE-O1-C1($o$, $I$, $R$, $pushPose$, $c_{0}$, $c_{1}$)

            \If{$c_{0} = \emptyset$ OR $c_{1} = \emptyset$}
              \State \textbf{continue}
            \EndIf
          \EndIf
        \EndIf

        \State $seq \gets 1$

        \State $translation \gets pushUnit * onePushDist * seq$

        \State $safeSweptArea \gets $GET-SAFE-SWEPT-AREA($o$, $translation$, $I$)

        \State $oSimPose \gets pushPose + translation$

        \State $suppC_{M} \gets $GET-OCC-COST(GET-OBS-POINTS($o$, $translation$), $occCostGrid$)

        \State $c_{3_{(Est)}} \gets \{oSimPose, R_{goal}\}$

        \State $C_{est} \gets ((c_{0} \neq \emptyset ? |c_{0}| : 0) + |c_{1}| + |c_{3_{(Est)}}|) * moveCost + |translation| * o.pushCost * suppC_{M}$

        \algstore{planforobstacle_merge_store}

  \end{algorithmic}
\end{algorithm}

\begin{algorithm}[H]

  \label{alg:07-custom-merge-optimized-planforobstacle-part2}

  \begin{algorithmic}[1]

        \algrestore{planforobstacle_merge_store}

        \While{$C_{est}$ $ \leq p_{opt}.cost$ AND $safeSweptArea \neq$ null}

          % \If{CHECK-NEW-OPENING($I.occGrid$, $o$, $translation$, $BA$)}
            \State $c_{2} \gets \{pushPose, oSimPose\}$
            \State $c_{3} \gets A$*($oSimPose$, $R_{goal}$, $I.withSimulatedObstacleMove$)
            \If{$c_{3} \neq \emptyset$}
              \State $p.components \gets$ $c_{0} \neq \emptyset$ ? [$c_{0}$, $c_{1}$, $c_{2}$, $c_{3}$] : [$c_{1}$, $c_{2}$, $c_{3}$]
              \State $p.cost \gets ((c_{0} \neq \emptyset ? |c_{0}| : 0) + |c_{1}| + |c_{3}|) * moveCost + |c_{2}| * o.pushCost * suppC_{M}$
              \State $p.minCost \gets |c_{2}| * o.pushCost * suppC_{M} + |c_{3}| * moveCost$
              \State $p.o \gets$ COPY($o$)
              \State $p.translation \gets translation$
              \State $p.safeSweptArea \gets safeSweptArea$
              \State $P_{o,d} \gets P_{o,d} \bigcup \{p\}$
              \If{$p.cost < p_{opt}.cost$}
                \State $p_{opt} \gets p$
              \EndIf
            \EndIf
          % \EndIf

          \State $seq \gets seq + 1$

          \State $translation \gets pushUnit * onePushDist * seq$

          \State $safeSweptArea \gets $GET-SAFE-SWEPT-AREA($o$, $translation$, $I$)

          \State $oSimPose \gets pushPose + translation$

          \State $suppC_{M} \gets $GET-OCC-COST(GET-OBS-POINTS($o$, $translation$), $occCostGrid$)

          \State $c_{3_{(Est)}} \gets \{oSimPose, R_{goal}\}$

          \State $C_{est} \gets ((c_{0} \neq \emptyset ? |c_{0}| : 0) + |c_{1}| + |c_{3_{(Est)}}|) * moveCost + |translation| * o.pushCost * suppC_{M}$

        \EndWhile

      \EndFor

    \State \textbf{return} $p \in P_{o,d}$ with minimal $p.cost$ or null if $P_{o,d} = \emptyset$

    \EndProcedure

  \end{algorithmic}
\end{algorithm}


\clearpage

\section{Efficient Opening Detection, Levihn M. and Stilman M. (2011), Commented}\label{appendix_eod_section}

\paragraph{Note on GET-$M'_{i}$-MATRIX}\label{get_mi_matrix_note} The world $W$ is represented by an occupancy grid: this procedure extends the obstacle $M_{i}$ by the robot's diameter, giving us $M'_{i}$, then represented as a binary matrix $M$, which has the size of the bounding box of $M'_{i}$.

% \paragraph{Note on absence of initially Blocking Areas}\label{no_ba_note} If there are no initially blocking areas, then it means there already are openings, and we must therefore return true. This limit case is not taken into account by the original algorithm formulation.

\paragraph{Note on GET-NEW-X/Y-POS}\label{x-y-pos_note} Simulate the set of manipulation actions $A_{M}$ and get the world coordinates of the object.

\paragraph{Note on interpreting the value of Z}\label{interpreting_z_note} If $Z$ is the 0-matrix, it means no new openings were detected, as all blocking areas still are blocking after the manipulation. Else, it means that one intersecting area has disappeared, meaning a possible new opening.

\paragraph{Note on detecting BAs}\label{check_blockage_note} Check for blockage between $M'_{i}$ (represented by $M$) and other obstacles (which data is contained in the occupancy grid $G$). For that we only call ASSIGN-NR if at least one of the two current elements of both matrices signals an obstacle ($\neq 0$).

\paragraph{Note on deletion}\label{deletion_note} If an index has already been deleted in an element of $BS$, delete it everywhere else because if part of a previous blocking area is detected, it means that the robot is still blocked by the same area. If for a same element of $BS$, there is an index in $BA[x][y] and BA^*_{s}[x][y]$, then it means that the blocking area still exists, thus we zero it in $BS$.

\begin{algorithm}[H]

  \caption{Efficient Local Opening Detection algorithm, Levihn et. al. (2011), commented}

  \label{alg:opening_detection}

  \begin{algorithmic}[1]

    \Procedure{CHECK-NEW-OPENING}{$G$, $M_{i}$, $A_{M}$, $BA$}

      \State $M \gets$ GET-$M'_{i}$-MATRIX($M_{i}$) \Comment{\nameref{get_mi_matrix_note}}
      \State $x_{offset} \gets M_{i}.x$ \Comment{$M_{i}.x$ and $M_{i}.y$ are the map coordinates of $M_{i}$'s top left corner.}
      \State $y_{offset} \gets M_{i}.y$
      \If{$BA$ is null} \Comment{BA needs not be recomputed if the environment did not change.}
        \State $BA \gets$ GET-BLOCKING-AREAS($x_{offset}$, $y_{offset}$, $M$, $G$) \Comment{Blocking areas before manip.}
      \EndIf
      \State $x_{offset} \gets$ GET-NEW-X-POS($A_{M}$, $M_{i}$) \Comment{\nameref{x-y-pos_note}}
      \State $y_{offset} \gets$ GET-NEW-Y-POS($A_{M}$, $M_{i}$)
      \State $BA_{s} \gets$ GET-BLOCKING-AREAS($x_{offset}$, $y_{offset}$, $M$, $G$) \Comment{Blocking areas after manip.}
      \State $BA^*_{s} \gets [0][0](dim(M))$ \Comment{Since the window of $BA_{s}$ shifted with the manipulation of $M_{i}$, ...}

      \For{$k$ from 0 to $|BA^*_{s}|$} \Comment{... we shift it back for the future comparison with $BA$.}
        \For{$l$ from 0 to $|BA^*_{s}[i]|$}
          \State $x \gets (x_{offset} - M_{i}.x) + k$
          \State $y \gets (y_{offset} - M_{i}.y) + l$
          \If{$0 < x < |BA^*_{s}|$ AND $0 < y < |BA^*_{s}[x]|$}
            \State $BA^*_{s}[x][y] \gets |BA^*_{s}|[k][l]$
          \EndIf
        \EndFor
      \EndFor

      \State $Z \gets$ COMPARE($BA$, $BA^*_{s}$) \Comment{Finally, compare the two blocking aread configurations.}
      \If{$Z = [0][0](dim(M))$} \Comment{\nameref{interpreting_z_note}}
        \State return $false$
      \EndIf
      \State return $true$

    \EndProcedure

    \\

    \Procedure{GET-BLOCKING-AREAS}{$x_{off}$, $y_{off}$, $M$, $G$}

    \State $index \gets 1$
    \State $BA \gets [0][0](dim(M))$ \Comment{$[0][0](dim(M))$ represents the 0-Matrix of dimensions = $dim(M)$}

    \For{$x$ from 0 to $|M|$} \Comment{Iterate over $M$ to detect and tag blocking areas.}
      \For{$y$ from 0 to $|M[x]|$}
        \If{$M[x][y] \neq 0$ AND $G[x+x_{off}][y+y_{off}] \neq 0$} \Comment{\nameref{check_blockage_note}}
          \State ASSIGN-NR($BA$, $x$, $y$, $index$)  \Comment{$BA$ and $index$ are directly modified in the call.}
        \EndIf
      \EndFor
    \EndFor

    \State return $BA$ \Comment{Return the saved information on the blocked areas.}

    \EndProcedure

    \algstore{opening_detection}

  \end{algorithmic}

\end{algorithm}


\clearpage

\begin{algorithm}[H]
  \begin{algorithmic}[1]
    \algrestore{opening_detection}

    \Procedure{ASSIGN-NR}{$BA$, $x$, $y$, $index$}

    \For{$i$ from -1 to 1} \Comment{Assignment is performed based on the 3*3 neighborhood.}
      \For{$j$ from -1 to 1}
        \If{$BA[x+i][y+j] \neq 0$} \Comment{If a number is already in the neighborhood, ...}
          \State $BA[x][y] \gets BA[x+i][y+j]$ \Comment{... the same number is assigned to the element.}
          \State \textbf{return}
        \EndIf
      \EndFor
    \EndFor

    \State $BA[x][y] \gets index$ \Comment{Else a new number is assigned, equal to the new index.}
    \State $index \gets index + 1$ \Comment{Only increment index if new intersection is created.}
    \State \textbf{return}

    \EndProcedure

    \\

    \Procedure{COMPARE}{$BA$, $BA^*_{s}$} \Comment{This function checks for non-zero entries in both matrices.}

    \State $BS \gets$ COPY($BA$)
    \State $del_{num} \gets \emptyset$ \Comment{Set of the indexes of obstacles to delete from $BS$.}

    \For{$x$ from 0 to $|BS|$} \Comment{Iterate over $BS$.}
      \For{$y$ from 0 to $|BS[x]|$}
        \If{$BA[x][y] \in del_{num}$} \Comment{\nameref{deletion_note}}
          \State $BS[x][y] \gets 0$
        \EndIf
        \If{$BA[x][y] \neq 0$ AND $BA^*_{s}[x][y] \neq 0$} \Comment{\nameref{deletion_note}}
          \State $del_{num} = del_{num} \bigcup BS[x][y]$
          \State $BS[x][y] \gets 0$
        \EndIf
      \EndFor
    \EndFor

    \State \textbf{return} $BS$

    \EndProcedure

  \end{algorithmic}

\end{algorithm}

