% Chapter 6

\chapter{Conclusion and Perspectives} % Main chapter title

\label{Chapter6} % For referencing the chapter elsewhere, use \ref{Chapter6}

\paragraph{} In this work, we presented a detailed state of the art of the NAMO domain, and provided a set of criterias that allowed us to compare the many propositions that exist in this domain, and situate our overall proposition in contrast to them. We concluded this analysis with a synthesis on the many different hypotheses and approaches present in the litterature, and highlighted problematics that have not yet been explored (dynamic and social navigation in particular).

\paragraph{} We then analyzed the state of the art algorithms for locally optimal NAMO in unknown environments \parencite{wu_navigation_2010, levihn_locally_2014}, fixed and reformulated them properly with pseudocode, and finally, built a new set of propositions on this foundation, that constitute the first steps toward bringing dynamic and social constraints to the domain. A first proposition modified the obtained foundation to make it fit with our hypotheses in regard to the use of the Pepper Robot in the context of the Robocup@Home challenge. A second proposition consisted in allowing the robot to take the action of identifying into consideration in its optimal plan computation, in order to allow the robot to then use this new information to know whether or not it is authorized to move the obstacle. A third proposition questioned the social acceptability of placing an obstacle in a specific spot and brought a basic way to integrate this new constraint. Finally, a fourth and last proposition exposes the minimal necessary changes to allow the algorithm to still make optimal plans in a dynamic environment.

\paragraph{} While we were not able to test all of them in simulation or reality at the time of this writing, we were able to validate our first proposition and the possibility for the Pepper robot current actuators to be used in the real application.

\paragraph{} The first perspective to our work is of course to validate then build upon our existing propositions, in order to manage more complex dynamic and social navigation models, but also to interact with other agents, be them humans or other robots. One can easily imagine robot interactions for NAMO where the robot kindly asks another agent to move out of its way, or ask help in moving obstacles that it is not capable of moving alone (which is a problem that has already been addressed in existing research, but only for the purpose of moving a user-specified object \parencite{amato_planning_2015}). Another interesting perspective is that we could also explore about the environment observation problem surrounding NAMO, and find an optimal way for a robot to use data provided by other agents (robots or IoT sensors), and ask for other agents to check the surroundings of an obstacle the robot is currently considering for movement. As the need for better performance may rise, we can also study ways of sacrificing the local optimality of our proposition in a controlled manner in order to improve performance as in \parencite{levihn_planning_2013}. In the same way, as we head toward real-world experimentations, we will maybe need to use approaches that better take uncertainty into account, as in \parencite{stilman_planning_2007, levihn_foresight_2013, levihn_planning_2013, scholz_navigation_2016}. Finally, evaluating this work in the actual context of the next Robocup@Home challenges would be a great way to validate it.

% \textbf{Link of Jilles about cooperating robots that push obstacles together: http://www.ccs.neu.edu/home/camato/videos.html.}
