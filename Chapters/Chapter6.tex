% Chapter 6

\chapter{Conclusion and Perspectives} % Main chapter title

\label{Chapter6} % For referencing the chapter elsewhere, use \ref{Chapter6}

La conclusion n'est pas le résumé de l'écrit,  mais la fin. Elle récapitule d'abord brièvement le cheminement de pensée et en particulier les conclusions intermédiaires décrites dans le développement. Puis elle énumère les contributions/réalisations.

La conclusion ne peut faire référence à des idées dont il n'a pas été question dans le développement. On ne saurait y trouver des faits nouveaux car la conclusion n'est en principe pas une ouverture sur d'autres idées; pour cela il est préférable d'ajouter un paragraphe "Perspectives".

La conclusion s'ouvre plutôt sur l'action et doit être formulée très clairement, sous peine d'en diminuer l'impact.
