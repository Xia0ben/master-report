% Chapter 6

\chapter{Conclusion and Perspectives} % Main chapter title

\label{Chapter6} % For referencing the chapter elsewhere, use \ref{Chapter6}

\section{Contributions}

\section{Future work}


% La conclusion n'est pas le résumé de l'écrit,  mais la fin. Elle récapitule d'abord brièvement le cheminement de pensée et en particulier les conclusions intermédiaires décrites dans le développement. Puis elle énumère les contributions/réalisations.

% La conclusion ne peut faire référence à des idées dont il n'a pas été question dans le développement. On ne saurait y trouver des faits nouveaux car la conclusion n'est en principe pas une ouverture sur d'autres idées; pour cela il est préférable d'ajouter un paragraphe "Perspectives".

% La conclusion s'ouvre plutôt sur l'action et doit être formulée très clairement, sous peine d'en diminuer l'impact.


%%% COPY PASTA
% \paragraph{} Que faire en cas de rencontre d’une personne (ne pas gêner, la contourner ou attendre son départ?), que faire en cas de rencontre d’un objet dans un passage mentionné libre dans la carte (le contourner, le déplacer ou mettre à jour la carte et replanifier?). L’ensemble de ces questions devra être abordé en tenant compte, d’une part, du respect des règles sociales des humains, et d’autre part, du besoin d’arriver au plus vite à la destination visée.
%
% \paragraph{} Le travail sera mené en s’appuyant sur des travaux existants dans l’équipe. En particulier pour les aspects cartographie (ROS-navigation), reconnaissance des humains et des objets (librairies de références, adaptations faites dans l’équipe et travaux de Christian Wolf) et navigation sociale (détection des comportements humains et modélisation des flux humains).
%%% END COPY PASTA


% \textbf{Link of Jilles about cooperating robots that push obstacles together: http://www.ccs.neu.edu/home/camato/videos.html.}
