% Chapter 1

\chapter{Introduction} % Main chapter title

\label{Chapter1} % For referencing the chapter elsewhere, use \ref{Chapter1}

\section{Problem}

% L'introduction situe le problème, l'expose, insiste sur son importance et indique la manière dont il est envisagé. A l'introduction est associée une  présentation préliminaire de la manière de traiter la question (méthode). L'introduction doit aussi exposer l’étude de l’existant dans le domaine précis qui concerne le PFE et faire ressortir la nécessité de réalisations complémentaires comme celle qui fait l'objet du PFE.

% La suite du document constitue un développement structuré en sections, avec autant de niveaux de subdivision qu’il est nécessaire:

% Vous y expliquerez le contexte du PFE, le problème traité, la démarche suivie, les résultats obtenus. Vous veillerez à bien préciser votre contribution personnelle dans le déroulement du projet global. Vous définirez soigneusement toutes les notations particulières que vous utiliserez. Vous présenterez des tableaux et des figures numérotés et référencés dans le texte et accompagnés de légendes.

\section{Existing approaches}
