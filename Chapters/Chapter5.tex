% Chapter 5

\chapter{Experimentations \& Validation} % Main chapter title

\label{Chapter5} % For referencing the chapter elsewhere, use \ref{Chapter5}

\section{On experimentation repeatability with ROS and Pepper}

% TODO Add text here about Dockerfile, why use it, and what my variation with Pepper brings. Propose a modification on ROS wiki article for Pepper that links to my Doccker file ? Keywords : Portable Linux ROS Workspace for Pepper with Docker.

\section{Pushing tests with Pepper}

% TODO Add text here about manipulation experiments done with Pepper and figures taken with Cylia and Lelio

\section{ROS-Standards Compatible Simulator}

\section{Simulation results}

% \paragraph{} Re-reading the three papers about NAMO in unknown environments also offered the opportunity to reconsider three criteria for evaluating our experiment, in addition to the overall runtime:
%
% \begin{itemize}
%   \item The average number of obstacle evaluations: characterizes the gain of ordering candidate objects for evaluation.
%   \item The average number of path planning algorithm calls: characterizes the gain of the choice of the bound during obstacle evaluation.
%   \item The most efficient move/manipulation ratio: characterizes the density of the environment for which the algorithm gives the best performance.
% \end{itemize}
